\documentclass{article}

\usepackage[margin=1.5cm]{geometry}

%http://filoxus.blogspot.com/2008/01/how-to-insert-watermark-in-latex.html
\usepackage{graphicx}
\usepackage{eso-pic}
\usepackage{color}
\makeatletter
\AddToShipoutPicture{%
            \setlength{\@tempdimb}{.5\paperwidth}%
            \setlength{\@tempdimc}{.5\paperheight}%
            \setlength{\unitlength}{1pt}%
            \put(\strip@pt\@tempdimb,\strip@pt\@tempdimc){%
        \makebox(10,-200){\rotatebox{45}{\textcolor[gray]{0.65}%
        {\Huge{DRAFT DRAFT DRAFT DRAFT DRAFT DRAFT}}}}%
            }%
        }
\makeatother


\title{Solving Low Order Polynomials by Roots}
\author{Perplexed Perspectives} 

\begin{document}
\maketitle

\section{Introduction}
This is a brief review of how to extract roots of quadratic,
cubic, and quartic equations.  The material is neither new
nor complete, but it is hoped to be pedagogically useful.
The section on the quadratic is a little long, but the
intention is to establish the basic ideas in the simple
context.

\section{The Quadratic}
The formula for solving
\begin{equation}\label{eq:standard-quadratic}
 a x^2 + bx + c = 0
\end{equation}
is well known, but we will take a slightly novel approach.
First, note that if $a=0$, this degenerates to a linear equation,
so we take the case $a\ne$, so we are looking for roots to
the corresponding monic polynomial, \emph{i.e.},
\begin{equation}\label{eq:monic-quadratic}
  x^2 + \frac{b}{a} x + \frac{c}{a} = 0.
\end{equation} 
This polynomial can be written as the product of its (unknown) roots
\footnote{To be explicit, suppose $r_1$ is a root. 
Then by synthetic division on can get 
\begin{equation}
  x^2 + (b/a)x + (c/a) = (x - r_1)P(x) + Q(x)
\end{equation}
where the degree of $P(x)$ is lower that of the original (quadratic)
polynomial and the degree of $Q(x)$
is less than that of the (linear) factor with ($Q(x)=0$ indentically
as a special case).  In this particular case, because the
factor $x-r_1$ is linear, $Q(x)$ must be a constant (possibly $0$),
and the assumption that $r_1$ is a root gives:
\begin{equation}
 Q(r_1) = 0 + Q(r_1) =  (r_1 - r_1)P(r_1) + Q(r_1) = 0.
\end{equation}
Again, since $x-r_1$ is linear, the degree of $P(x)$ must be one
less than the original polynomial, in this case linear, and further
$P(x)$ to make the factorization work.  Thus one can write
$P(x) = x - r_2$ for some (unknown) $r_2$.},
$r_1$ and $r_2$ , so 
\begin{equation}
  x^2 + \frac{b}{a} x + \frac{c}{a} = (x - r_1) (x - r_2)
\end{equation}
 Note that this is
an equality of polynomials, so that
\begin{eqnarray}
 r_1 + r_2 & = & -b/a 
\label{eq:symmetric-quadratic1} \\
 r_1 r_2 & =  & c/a.
\label{eq:symmetric-quadratic2}
\end{eqnarray}
From a more elementary level, one can see that if
$r_1$ and $r_2$ satisfy eqs~\ref{eq:symmetric-quadratic1} 
and~\ref{eq:symmetric-quadratic2}, then each is a root
of \ref{eq:standard-quadratic} (and indeed, the only roots, 
again by syntehetic division)\footnote{Again, being very explicit,
obvious $r_1$ and $r_2$ are roots of $(x-r1)(x-r2)$, and 
eqs~(\ref{eq:symmetric-quadratic1} and~\ref{eq:symmetric-quadratic2} imply
that this polynomial is the same as the one to be solved (up to a 
factor of $a$).}.
Now, knowing $r_1 + r_2$, it is clear that we would be done
if we knew $r_1 - r_2$, because then we would just need to
solve two linear equations in two unknowns.  Just to establish
notation, if we have
\begin{eqnarray}
s_1 & = & r_1 + r_2 \\
s_2 & = & r_1 - r_2 \\
\end{eqnarray}
then knowledge of $s_1$ and $s_2$ would allow one to solve
or $r_1$ and $r_2$.
Also, we display the elementary symmetric polynomials of
the roots:
\begin{eqnarray}
e_1 & = & r_1 + r_2 = -b/a \\
e_2 & = & r_1 r_2 = c/a.
\end{eqnarray}

Now we are ready for the trick.  Note that $r_1 - r_2$ is not symmetric
in the variables $r_1$ and $r_2$ (indeed, it's antisymmetric, \emph{i.e.},
what the values of $r_1$ and $r_2$ results in the negative of the
original value).  However, $(r_1 - r_2)^2$ is symmetic in the variables,
and indeed
\begin{eqnarray}
\left( r_1 - r_2 \right)^2 & = & r_1^2 - 2 r_1 r_2 + r_2^2 \\
 & = & \left( r_1 + r_2 \right)^2 - 4 r_1 r_2 \\ 
 & = & e_1 ^ 2 - 4 e_2.
\end{eqnarray}
So now we know that
\begin{eqnarray}
  r_1 + r_2 & = & s_1 = -b/a \\
  r_1 - r_2 & = & s_2 = \sqrt{(b/a)^2 - 4(c/a)}.
\end{eqnarray}
Now we have two linear equations which can directly solved.
There is a choice of sign in taking the square root, but
changing that sign corresponds to swapping the values of
$r_1$ and $r_2$.  Thus by taking the sum and difference
of the above equations, one gets
\begin{equation}
  r_1, r_2 = \frac{-b \pm \sqrt{b^2 - 4 a c}}{2a}
\end{equation}

\section{The Cubic}
Now, with only a little bit of cleverness, one can solve the
cubic (thisis the \lq\lq Lagrange method\rq\rq{}).
First, let's write the cubic in the following monic form:
\begin{equation}
  x^3 - e_1 x^2 + e_2 x - e_3 = 0
\end{equation}
The signs are chosen so that the the coefficients are the
elementary symmetric polynomials in terms of the 
(unknown) roots ($r_1$, $r_2$, $r_3$):
\begin{eqnarray}
 e_1 & = & r_1 + r_2 + r_3 \\
 e_2 & = & r_1 r_2 + r_2 r_3 + r_3 r_1 \\
 e_3 & = & r_1 r_2 r_3 
\end{eqnarray}

Now, we come to the one part which requires additional cleverness.
Let $\zeta$ be a primitive third root of unit\footnote{That is,
$\zeta = e^{2\pi i/3}$.  The other root, 
$\zeta = e^{4\pi i/3}$, would also work.}, \emph{i.e.}:
\begin{equation}
 \zeta = -\frac{1}{2} + \frac{\sqrt{3}}{2}i.
\end{equation}

Now we take 
\begin{eqnarray}
s_1  =  r_1  & + r_2 & +  r_3 \label{eq:cube-lin1}\\
s_2  =  r_1  &  + \zeta r_2 & +  \zeta^2 r_3 \label{eq:cube-lin2} \\
s_3  =  r_1  & + \zeta^2 r_2 & +  \zeta r_3 \label{eq:cube-lin3} \\
\end{eqnarray}

Let's take a moment to consider how the $s_i$'s are affected
in we permute the roots.  First, consider the cyclic rotation
\begin{equation}
  r_1 = \tilde r_2 ,
  r_2 = \tilde r_3 ,
  r_3 = \tilde r_1 
\end{equation}
Then
\begin{eqnarray}
\tilde s_1 & = & \tilde r_1 + \tilde r_2 + \tilde r_3 \\
       & = & r_3 + r_1 + r_2 \\
       & = & s_1
\end{eqnarray}
and
\begin{eqnarray}
\tilde s_2 & = & \tilde r_1 + \zeta \tilde r_2 +
                       \zeta^2 \tilde r_3 \\
         & = & r_3 + \zeta \tilde r_1 +
                       \zeta^2 \tilde r_2 \\
         & = \zeta s_3
\end{eqnarray}
and similarly
\begin{equation}
  \tilde s_3 = \zeta^2 s_3.
\end{equation}
Similarly, swapping $r_2$ and $r_3$ correspond to 
swapping $s_2$ and $s_3$.  This is sufficient to 
prove\footnote{The symmetric group on $n$ objects
is generated by an $n$ cycle  (12\ldots n) and 
and a two cycle (12)}that the quantities $s_2 s_3$ and $s_2^3 + s_3^3$
are symmetric in the roots.
Explicityly calculations (recalling that $\zeta$ satisfies
$\zeta^2 + \zeta + 1 = 0$) give
\begin{eqnarray}
  s_2 s_3 & = & r_1^2 + r_2^2 + r_3^2 - r_1 r_2 - r_2 r_3 - r_3 r_1\\
     & = & e_1^2 - 3 e_2\\
     & = & \omega_1
\end{eqnarray}
and
\begin{eqnarray}
 s_2 ^3 + s_3^3 & = & 2 r_1^3 + 2 r_2^3 + 2 r_3^3 - 3 r_1 r_2 r_3 \\
   & = & 2 e_1^3 - 6 e_1 e_2 - 9 e_3 \\
   & = & \omega_2.
\end{eqnarray}
where $\omega_1$ and $\omega_2$ have been introduced to summarize
the expressions involving the $e_i$'s.

Thus, we know the sum and product of $s_2^3$ and $s_3^3$.  That is,
$s_2^3$ and $s_3^3$ are the roots of a quadratic whose cooeficients
can be expressed in term of the the coefficients of the original
cubic. Explicitly,
\begin{equation}
 s_2^3, s_3^3 = \frac{\omega_1 \pm \sqrt{\omega_1^2 - 4\omega_2}}{2}
\end{equation}

As $s_1 = e_1$ is simply the (negative of) the linear coefficient,
once one has $s_2$ and $s_3$, one can immediately solve for the
roots 
\begin{eqnarray}
  r_1 & = & \frac{1}{3} \left( s_1 + s_2 + s_3 \right) 
                 \label{eq:cubic-root-1} \\
  r_2 & = & \frac{1}{3} \left( s_1 + \zeta^2 s_2 + \zeta s_3 \right) 
                 \label{eq:cubic-root-2}  \\
  r_3 & = & \frac{1}{3} \left( s_1 + \zeta s_2 + \zeta^2 s_3 \right) 
                 \label{eq:cubic-root-3}\\
\end{eqnarray}
Note that, in terms of the coefficients of the polynomial whose
solution is sought, we know $s_2^3 + s_3^3$ and,
as we know the product $s_2 s_3$, we know the product $s_1^3 s_2^3$.
Thus we can solve a quadratic to get the values of
$s_2^3$ and $s_3^3$.  In solving this quadratic, there is a
choice of which root cooresponds to $s_2^3$ and which corresponds
to $s_3^3$.  However, swapping $s_2$ and $s_3$ corresponds precisely
to swapping $r_2$ and $r_3$ in 
equations~\ref{eq:cubic-root-1}--\ref{eq:cubic-root-3}.
Further, after solving the quadratic, we have $s_2^3$ and
$s_3^3$, so the cube roots of each are required.  In general,
each will have three distinct cube roots.  However, we actually
know the product $s_2 s_3$, so if one chooses one of the cube
roots, the other is forced.  Very explicitly, if
one replaces $s_2$ with $\zeta s_2$, then one must replace
$s_3$ with $\zeta^2 s_3$, and in 
equations~\ref{eq:cubic-root-1}--\ref{eq:cubic-root-3}.
the roots of the cubic are cyclically permuted.



\end{document}
